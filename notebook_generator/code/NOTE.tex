## Fibonacci Number:
gcd(F_m, F_n) = F_gcd(m, n)

## Lucas Number:
L(n) = L(n-1) + L(n-2); L(0)=2, L(1)=1
- Number of edge cover of a cycle graph C_n is L_n

## Catalan Number:
C(n+1) = C(0)C(n) + C(1)C(n-1) + ... + C(n)C(0)
C(n) = nCr(2n, n) - nCr(2n, n + 1)
C(n) = nCr(2n, n) / (n + 1)

- The number of valid bracket sequences of length 2n when a prefix of the sequence is given (among f are first bracket and s are second bracket)
C(2n, f, s) = ncr(2n-(f+s), n-f) - ncr(2n-(f+s), (n-f)+(f-s)+1)

## Stirling Number of second Kind:
"Number of ways to partition a set of n labelled objects into k nonempty unlabelled subsets"

S(n, k) = S(n-1, k-1) + k * S(n-1, k)
S(n, 1) = 1, S(n, n) = 1

- Time Complexity: \( O(k \log n) \)
\verb|
ll get_sn2(int n, int k) {
  ll sn2 = 0;
  for (int i = 0; i <= k; ++i) {
    ll now = nCr(k, i)*powmod(k - i, n, mod)%mod;
    if (i&1)  now = now * (mod - 1) % mod;
    sn2 = (sn2 + now) % mod;
  }
  sn2 = sn2 * ifact[k] % mod;
  return sn2;
}
|
- Number of ways to color a 1*n grid with k colors such that each color is used at least once = k! * sn2(n,k)

## Derangement:
"A derangement is a permutation of the elements of a set, such that no element appear in its original position."
1, 0, 1, 2, 9, 44, 265, ...
D(n) = (n-1) (D(n-1) + D(n-2)) = n * D(n-1) + (-1)^n
D(0) = 1, D(1) = 0

## Partition number:
for (int i = 1; i <= n; ++i) {
  pent[2 * i - 1] = i * (3 * i - 1) / 2;
  pent[2 * i] = i * (3 * i + 1) / 2;
}
p[0] = 1;
for (int i = 1; i <= n; ++i) {
  p[i] = 0;
  for (int j = 1, k = 0; pent[j] <= i; ++j) {
    if (k < 2) p[i] = add(p[i], p[i - pent[j]]);
    else p[i] = sub(p[i], p[i - pent[j]]); ++k, k &= 3;
  }
}

## Ballot Theorem:
Suppose that in an election, candidate A receives
a votes and candidate B receives b votes, where a ≥ kb for some positive
integer k. Compute the number of ways the ballots can be ordered so that
A maintains more than k times as many votes as B throughout the counting
of the ballots.
The solution to the ballot problem is ((a − kb)/(a+b)) * C(a+b, a)

## Classical Problem
F(n, k) = number of ways to color n objects using exactly k colors

Let G(n, k) be the number of ways to color n objects using no more than k colors.

Then,
F(n, k) = G(n, k) - C(k, 1) * G(n, k-1) + C(k, 2) * G(n, k-2) - C(k, 3) * G(n, k-3) ...

Determining G(n, k) :

Suppose, we are given a 1 * n grid. Any two adjacent cells can not have same color.
Then, G(n, k) = k * ((k-1)^(n-1))

If no such condition on adjacent cells.
Then, G(n, k) = k ^ n

/*
1/(1 - x) = 1 + x + x^2 + x^3 + ... ... ...
1/(1 - ax) = 1 + ax + (ax)^2 + (ax)^3 + ... ... ...
1/(1 - x)^2 = 1 + 2x + 3x^2 + 4x^3 + ... ... ...
1/(1 - x)^3 = C(2, 2) + C(3, 2)x + C(4, 2)x^2 + C(5, 2)x^3 + ... ... ...
1/(1 - ax)^(k + 1) = 1 + C(1 + k, k)(ax) + C(2 + k, k)(ax)^2 + C(3 + k, k)(ax)^3 + ... ... ...
x(x + 1)(1 - x)^-3 = 1 + x + 4x^2 + 9x^3 + 16x^4 + 25x^5 + ... ... ...
e^x = 1 + x + (x^2)/2! + (x^3)/3! + (x^4)/4! + ... ... ... 
*/

/* Extended Binomial Theorem
C(-n, r) = C(n + r - 1, r) * (-1)^r // n >= 0 and r >= 0
C(n, r) = (n(n - 1)(n - 2) ... (n - (r - 1))) / r! // works for any n
(x + a)^-n = a^(-n) + C(-n, 1) x^1 a^(-n-1) + C(-n, 2) x^2 a^(-n-2) + C(-n, 3) x^3 a^(-n-3) + ... ... ...
*/

\[ \sum_{i = 1}^n \gcd(i, n) = \sum_{d | n} d\phi(\frac{n}{d}) \]

\( 1^4+2^4+3^4+...+n^4=\frac{n(n+1)(2n+1)(3n2+3n+-1)}{30} \) \\
\( S_{(a, n)} = 1^a + 2^a + 3^a + 4^a + ... + n^a \) \\
\( S_{(a, n)} = \frac{1}{a+1} [ n^{a+1}-(-1)^a + \sum_{i=2}^a (-1)^i \binom{a+1}{i}S_{(a-i+1, n)} ] \) \\
\( 1.2 + 2.3 + 3.4 + ... = \frac{1}{3} n(n+1)(n+2) \) \\
\( \sum_{i=1}^n f_k(i) = \frac{1}{k+1} n(n+1)(n+2)...(n+k) = \frac{1}{k+1}\frac{(n+k)!}{(n-1)!} \) \\
\( \sum_{i=0}{n}ix^i = 1 + 2x^2 + 3x^3 + 4x^4 + 5x^5+ ... +nx^n = \frac{(x-(n+1)x^{n+1}+nx^{n+2})} {(x-1)^2} \) \\

## Condiational Probability:
\[ P(A|B) = \frac{P(A \cap B)}{P(B)} \]
## Bayes' Theorem:
\[ P(A|B) = \frac{P(B|A)P(A)}{P(B)} \]